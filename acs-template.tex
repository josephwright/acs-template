%%%%%%%%%%%%%%%%%%%%%%%%%%%%%%%%%%%%%%%%%%%%%%%%%%%%%%%%%%%%%%%%%%%%%
%% This is a "template" model document for submission to the
%% American Chemical Society (ACS).
%%
%% The guidance here contains information about how you may wish to
%% modify it to match the requirements of various ACS journals. The
%% ACS do *not* typeset accepted articles using LaTeX, so there is
%% no specific class required.
%%
%% This template deliberately does *not* seek to reproduce
%% the layout of the typeset journal: this is explicitly not
%% required by the ACS for LaTeX submissions.
%%
%% Please report any issues with the template at
%% https://github.com/josephwright/acs-template/issues
%%
%% Released under the Creative Commons 0 license
%% https://creativecommons.org/public-domain/cc0/
%% 
%% Copyight (c) 2025 Joseph Wright
%%%%%%%%%%%%%%%%%%%%%%%%%%%%%%%%%%%%%%%%%%%%%%%%%%%%%%%%%%%%%%%%%%%%%
\documentclass[letterpaper]{article}

%%%%%%%%%%%%%%%%%%%%%%%%%%%%%%%%%%%%%%%%%%%%%%%%%%%%%%%%%%%%%%%%%%%%%
%% Font setup - delete if you are using LuaLaTeX
%%%%%%%%%%%%%%%%%%%%%%%%%%%%%%%%%%%%%%%%%%%%%%%%%%%%%%%%%%%%%%%%%%%%%
\usepackage[T1]{fontenc}

%%%%%%%%%%%%%%%%%%%%%%%%%%%%%%%%%%%%%%%%%%%%%%%%%%%%%%%%%%%%%%%%%%%%%
%% Adjust the margins and allow for line spacing
%%%%%%%%%%%%%%%%%%%%%%%%%%%%%%%%%%%%%%%%%%%%%%%%%%%%%%%%%%%%%%%%%%%%%
\usepackage{geometry}
\geometry{margin = 1in}
\usepackage{setspace}

%%%%%%%%%%%%%%%%%%%%%%%%%%%%%%%%%%%%%%%%%%%%%%%%%%%%%%%%%%%%%%%%%%%%%
%% Reference support
%%
%% The recommended method for producing the reference section is
%% to use biblatex. If you wish to use a classical BibTeX
%% approach, this is easiest to achieve using the achemso package.
%% In that case, you should remove the biblatex lines.
%%%%%%%%%%%%%%%%%%%%%%%%%%%%%%%%%%%%%%%%%%%%%%%%%%%%%%%%%%%%%%%%%%%%%
% You can adjust the printing of DOI, article title, etc. using
% package options, e.g. "doi = true"; see the biblatex manual for
% details of adjusting the number of authors printed, e.g.
% "maxnames = 15" to print no more than 15 authors.
\usepackage[style = chem-acs]{biblatex}
\addbibresource{acs-template.bib}
% If you are using classical BibTeX, remove the above lines and 
% uncomment:
%\usepackage{achemso}

%%%%%%%%%%%%%%%%%%%%%%%%%%%%%%%%%%%%%%%%%%%%%%%%%%%%%%%%%%%%%%%%%%%%%
%% Graphic inclusion and scheme and chart support
%%%%%%%%%%%%%%%%%%%%%%%%%%%%%%%%%%%%%%%%%%%%%%%%%%%%%%%%%%%%%%%%%%%%%
\usepackage{graphicx}
\usepackage{float}
\newfloat{scheme}{htbp}{los}
\floatname{scheme}{Scheme}
\floatname{chart}{Chart}
\newfloat{graph}{htbp}{loh}

%%%%%%%%%%%%%%%%%%%%%%%%%%%%%%%%%%%%%%%%%%%%%%%%%%%%%%%%%%%%%%%%%%%%%
%% Common support packages
%%%%%%%%%%%%%%%%%%%%%%%%%%%%%%%%%%%%%%%%%%%%%%%%%%%%%%%%%%%%%%%%%%%%%
\usepackage{chemformula} % Formulas using \ch{}
% or
\usepackage[version = 4]{mhchem} % Formulas using \ce{}

%%%%%%%%%%%%%%%%%%%%%%%%%%%%%%%%%%%%%%%%%%%%%%%%%%%%%%%%%%%%%%%%%%%%%
%% Many journals require that sections are unnumbered: this 
%% is activated here
%%%%%%%%%%%%%%%%%%%%%%%%%%%%%%%%%%%%%%%%%%%%%%%%%%%%%%%%%%%%%%%%%%%%%
\setcounter{secnumdepth}{-1}

%%%%%%%%%%%%%%%%%%%%%%%%%%%%%%%%%%%%%%%%%%%%%%%%%%%%%%%%%%%%%%%%%%%%%
%% Place any additional macros here.  Please use \newcommand* where
%% possible, and avoid layout-changing macros (which are not used
%% when typesetting).
%%%%%%%%%%%%%%%%%%%%%%%%%%%%%%%%%%%%%%%%%%%%%%%%%%%%%%%%%%%%%%%%%%%%%
\newcommand*\mycommand[1]{\texttt{\emph{#1}}}

%%%%%%%%%%%%%%%%%%%%%%%%%%%%%%%%%%%%%%%%%%%%%%%%%%%%%%%%%%%%%%%%%%%%%
%% Author and title data:
%% the authblk package is currently the simplest way to provide this
%%%%%%%%%%%%%%%%%%%%%%%%%%%%%%%%%%%%%%%%%%%%%%%%%%%%%%%%%%%%%%%%%%%%%
\usepackage{authblk}
\author[1]{Andrew N. Other}
\author[1]{Fred T. Secondauthor}
\author[1]{I. Ken Groupleader*}
\affil[1]{Department of Chemistry, Unknown University, Unknown Town}
\author[2]{Susanne K. Laborator}
\affil[2]{Lead Discovery, BigPharma, Big Town, USA}
\author[1]{Kay T. Finally}

\title{A ``template'' model document for submission to the
  American Chemical Society (ACS)}
% Use the \date command for email address(s) of corresponding authors
\date{*Email: i.k.groupleader@unknown.uu}

\begin{document}

\maketitle

\begin{abstract}
  This is an example document for creating \LaTeX{} submissions to the American
  Chemical Society (ACS) for publication. As ACS does not use \LaTeX{} for
  typesetting accepted manuscripts, this template does not seek to
  reproduce the appearance of a published paper.
\end{abstract}

\section*{Keywords}

Some journals require keywords: these normally should be given immediately
after the abstract.

\section*{Abbreviations}

Some journals require a list of abbreviations: these normally should be given
immediately after the keyswords (if required).

%%%%%%%%%%%%%%%%%%%%%%%%%%%%%%%%%%%%%%%%%%%%%%%%%%%%%%%%%%%%%%%%%%%%%
%% Start the main part of the manuscript here.
%%%%%%%%%%%%%%%%%%%%%%%%%%%%%%%%%%%%%%%%%%%%%%%%%%%%%%%%%%%%%%%%%%%%%
\section{Introduction}

This is a paragraph of text to fill the introduction of the demonstration file. 

\section{Results and discussion}

\subsection{Outline}

The document layout should follow the style of the journal concerned. Where
appropriate, sections and subsections should be added in the normal way.

\subsection{References}

References should be given in the normal way in \LaTeX{}. If you are using
\textsf{biblatex} (as recommended) then you can use the full range of citation
commands it provides. If you choose to use classical Bib\TeX{}, the
\textsf{natbib} package will be loaded and you can use it's commands.

\subsection{Floats}

New float types are set up in the preamble. The means graphics are included as
follows (Scheme~\ref{sch:example}). As illustrated, the float is ``here'' if
possible.
\begin{scheme}
  \centering
  Your scheme graphic would go here: PDF graphics are recommended.
  %\includegraphics{graphic}
  \caption{An example scheme}
  \label{sch:example}
\end{scheme}

\begin{figure}
  \centering
  A standard figure environment.
  \label{fgr:example}
\end{figure}

The use of the different floating environments is not required, but it is
intended to make document preparation easier for authors. In general, you
should place your graphics where they make logical sense; the production
process will move them if needed.

\subsection{Math}

If packages such as \textsf{amsmath} are required, they should be loaded in the
preamble. However, the basic \LaTeX\ math(s) input should work correctly
without this. Some inline material $1 + 1 = 2$ followed by some display. \[ A =
\pi r^2 \]

It is possible to label equations in the usual way (Eq.~\ref{eqn:example}).
\begin{equation}
  \frac{\mathrm{d}}{\mathrm{d}x} \, r^2 = 2r \label{eqn:example}
\end{equation}
This can also be used to have equations containing graphical content. To align
the equation number with the middle of the graphic, rather than the bottom, a
minipage may be used.
\begin{equation}
  \begin{minipage}[c]{0.80\linewidth}
    \centering
    As illustrated here, the width of \\
    the minipage needs to allow some  \\
    space for the number to fit in to.
    %\includegraphics{graphic}
  \end{minipage}
  \label{eqn:graphic}
\end{equation}

\section{Experimental}

The usual experimental details should appear here. This could include a table,
which can be referenced as Table~\ref{tbl:example}. Notice that the caption is
positioned at the top of the table.
\begin{table}
  \caption{An example table}
  \label{tbl:example}
  \centering
  \begin{tabular}{ll}
    \hline
    Header one  & Header two  \\
    \hline
    Entry one   & Entry two   \\
    Entry three & Entry four  \\
    Entry five  & Entry five  \\
    Entry seven & Entry eight \\
    \hline
  \end{tabular}
\end{table}

Adding notes to tables can be complicated. Perhaps the easiest method is to
generate these using the basic \texttt{\textbackslash textsuperscript} and
\texttt{\textbackslash emph} macros, as illustrated (Table~\ref{tbl:notes}).
\begin{table}
  \caption{A table with notes}
  \label{tbl:notes}
  \centering
  \begin{tabular}{ll}
    \hline
    Header one                            & Header two \\
    \hline
    Entry one\textsuperscript{\emph{a}}   & Entry two  \\
    Entry three\textsuperscript{\emph{b}} & Entry four \\
    \hline
  \end{tabular}

  \textsuperscript{\emph{a}} Some text;
  \textsuperscript{\emph{b}} Some more text.
\end{table}

The example file also loads the optional \textsf{chemformula} and
\textsf{mhchem} packages, so that formulas are easy to input:
\texttt{\textbackslash ce\{H2SO4\}} gives \ce{H2SO4}. The two have similar
syntax but authors may prefer one or the other.

The use of new commands should be limited to simple things which will not
interfere with the production process. For example, \texttt{\textbackslash
mycommand} has been defined in this example, to give italic, mono-spaced text:
\mycommand{some text}.

\section*{Acknowledgements}

Please use ``The authors thank \ldots'' rather than ``The authors would like to
thank \ldots''.

\section*{Supporting information}

A listing of the contents of each file supplied as Supporting Information
should be included. For instructions on what should be included in the
Supporting Information as well as how to prepare this material for
publications, refer to the journal's Instructions for Authors.

The following files are available free of charge.
\begin{itemize}
  \item Filename-1: brief description
  \item Filename-2: brief description
\end{itemize}

%%%%%%%%%%%%%%%%%%%%%%%%%%%%%%%%%%%%%%%%%%%%%%%%%%%%%%%%%%%%%%%%%%%%%
%% If you are using classical BibTeX rather than biblatex,
%% remove the \printbibliography and uncomment the \bibliograpy one
%%%%%%%%%%%%%%%%%%%%%%%%%%%%%%%%%%%%%%%%%%%%%%%%%%%%%%%%%%%%%%%%%%%%%
\printbibliography
%\bibliography{acs-template.bib}

\newpage

\rule{0.05in}{1.75in}%
\begin{minipage}[b][1.75in]{3.25in}
  \sffamily
  \frenchspacing

  Some journals require a graphical entry for the Table of Contents. This
  should be laid out ``print ready'' so that the sizing of the text is correct.

  The space available depends on the journal: J. Am. Chem. Soc. allows 3.25 in
  by 1.75 in and requires sanserif text. Some journals want different sizes:
  you can easily adjust here.
  
  The two rules either side of the content are there to help judge the height
  of your material: they may be deleted once not required.
  
\end{minipage}%
\rule{0.05in}{1.75in}

\end{document}
